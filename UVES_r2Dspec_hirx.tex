\documentclass[12pt]{article}

\newcommand{\Npix}{\hbox{$N_{\rm pix}$}}
\newcommand{\Nord}{\hbox{$N_{\rm ord}$}}
\newcommand{\Nx}{\hbox{$N_{\rm x}$}}
\newcommand{\No}{\hbox{$N_{\rm o}$}}

\begin{document}

\section{Wavelength scale for HIRES\_REDUX files}

A 2D Legendre polynomial fit to the ThAr exposure provides a 2D
wavelength solution as a function of pixel coordinate in the
dispersion direction, $x$, and echelle diffraction order, $o$, of the
form
\begin{equation}
\lambda(x,o) = \frac{1}{o} \sum_{m=0}^{\Nx-1} \sum_{n=0}^{\No-1} C_{mn}P_m(x^\prime)P_n(o^\prime)\,,
\end{equation}
where $P_m(x^\prime)$ and $P_n(o^\prime)$ are the $m$th and $n$th
coefficients for the Legendre polynomials of order $\Nx$ and $\No$
respectively. The normalized values of $x$ and $o$ are used here:
\begin{equation}
x^\prime\equiv2(x-a_0)/a_1~~~~~\&~~~~~o^\prime\equiv2(o-b_0)/b_1\,,
\end{equation}
where the normalizing factors, $a$ and $b$, are fairly arbitrary.

It is convenient in the coding of UVES\_popler to only carry around
coefficients for the wavelength solution particular to each echelle
order, $o$,
\begin{equation}
A_m(o) \equiv \frac{1}{o} \sum_{n=0}^{\No-1} C_{mn}P_n(o^\prime)
\end{equation}
so that the wavelength anywhere along that order can be calculated
quickly and without reference to the coefficients relevant to other
orders,
\begin{equation}
\lambda(x; o) = \sum_{m=0}^{\Nx-1} A_m(o)P_m(x^\prime)\,.
\end{equation}

\end{document}
